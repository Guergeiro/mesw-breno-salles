Refactoring is the process of modifying the internal structure of an existing
codebase without changing its external behaviour \cite{becker1999refactoring}.
When migrating from a monolithic architecture to a microservices architecture,
it may be necessary to refactor the existing codebase to break it into
independent microservices. This can be a complex and time-consuming process,
particularly for large, complex systems \cite{newman2019monolith}.

There are several factors to consider when refactoring an existing codebase for
a microservices architecture \cite{newman2019monolith}. One challenge is
ensuring that the code is modular and loosely coupled so it can be developed
and deployed independently as a microservice. This may require restructuring
the code, introducing new abstractions and interfaces, and potentially even
rewriting parts of the code.

Another challenge is preserving the application's existing functionality while
making changes to the codebase. It is important to carefully plan and test the
refactoring process to ensure that the application continues to work as
expected after the changes are made.

Overall, refactoring an existing codebase for a microservices architecture can
be a significant undertaking, and it is important to carefully evaluate the
resources and time required to complete the process \cite{newman2019monolith}.
