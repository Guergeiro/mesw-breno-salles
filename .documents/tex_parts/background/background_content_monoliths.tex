Monolithic applications are software systems that are designed as a single,
self-contained unit \cite{gos2020comparison}. In other words, monolithic
applications are composed of a single, integrated codebase that includes all of
the necessary components and features for the application to run
\cite{kazanavivcius2019migrating}. This means that all of the different parts
of the application, such as the user interface, business logic, and database
access are all contained within a single codebase and are not modularized or
separated into distinct components that are separately deployed and executed.

Monolithic architecture is a traditional approach to software development that
has been widely used for many years \cite{gos2020comparison}. It is generally
characterized by a strong emphasis on simplicity and ease of development.
However, monolithic applications can also be more difficult to maintain and
update, as changes to one part of the codebase can have unintended consequences
on other parts of the system. This can make it challenging to introduce new
features or make changes to the application without significant testing and
debugging \cite{kazanavivcius2019migrating}.

Despite these challenges, monolithic applications are still widely used in many
contexts due to their simplicity and ease of development. They are particularly
well-suited for small to medium-sized applications that do not require a high
level of modularity or separation of concerns.
