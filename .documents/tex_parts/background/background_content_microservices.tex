Microservices is an architectural style that structures an application as a
collection of loosely coupled services \cite{newman2021building}. This means
that each microservice is a self-contained unit of functionality, which
communicates with other microservices through well-defined interfaces,
typically using a lightweight messaging protocol such as HTTP
\citeWEB{fowler-microservices}.

One key benefit of this approach is that it allows for greater flexibility and
scalability \cite{newman2019monolith}. Because each microservice is independent
and modular, it can be modified and deployed independently of the other
services in the application. This can make it easier to make changes to the
system, as it is not necessary to redeploy the entire application every time a
change is made. In addition, the modular nature of microservices allows for
easier scaling, as individual services can be scaled up or down as needed to
meet changing demand
\cite{newman2019monolith,newman2021building}\citeWEB{fowler-microservices}.

Another advantage of microservices is that they can be developed and maintained
by small, autonomous teams \cite{chen2018microservices}. This can be beneficial
for organizations with a large codebase or a distributed development team, as
it allows for more focused development and faster deployment of changes
\cite{nadareishvili2016microservice}.

However, there are also challenges to consider when adopting a microservices
architecture \citeWEB{fowler-microservices-tradeoffs}. One challenge is the
added complexity of managing a distributed system, as there may be a larger
number of moving parts to monitor and troubleshoot \cite{newman2021building}.
In addition, the communication between microservices can add latency to the
system, which may impact the performance of the overall application
\citeWEB{fowler-microservices-tradeoffs}\cite{pautasso2017microservices}.

Overall, microservices can be an effective way to structure an application,
particularly for large, complex systems that require a high degree of
flexibility and scalability \cite{newman2021building}. However, it is important
to carefully evaluate the trade-offs and consider whether the benefits of a
microservices architecture are worth the added complexity
\citeWEB{fowler-microservices-tradeoffs}.
