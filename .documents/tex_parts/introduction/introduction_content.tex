Microservices is an architectural style that evolved from Service Oriented
Architecture (SOA). Just like SOA, microservices are an alternative to
monolithic architecture. The main contrasts are that, while monolithic
applications are software systems with a single, integrated codebase that
includes all necessary components, and features
\cite{kazanavivcius2019migrating}, microservices tend to be separated, and
loosely coupled \cite{newman2021building}. Also while monoliths tend to be
easier to develop they may scale poorly and are harder to maintain when
compared to microservices \cite{newman2019monolith}. Microservices are
increasingly being used in the development of modern applications, particularly
in the areas of cloud computing \cite{balalaie2016migrating}. Many
organizations, including large enterprises and startups, are adopting
microservices as a way to build and deploy applications more quickly and
efficiently \citeWEB{richardson-microservices}. Microservices are particularly
well-suited for distributed, cloud-based environments, where they can take
advantage of the flexibility and scalability of the cloud
\citeWEB{fowler-microservices-prerequisites}. This type of architecture is
already being applied in multiple well-known companies, like Uber, Netflix,
eBay \citeWEB{microservices-users}\cite{ren2018migrating}, and also being
followed by the rest of the herd when compared to monolith architecture
\cite{taibi2017processes}.

Refactoring from monoliths to microservices is a heavily debated topic both in
the academic world and the industry. The main takes from this debate are that
refactoring is difficult and time-consuming, and companies struggle with
migrating their already existing monolithic applications to microservices
\cite{kamimura2018extracting}. To help address this, some tools were developed
that help with the refactor \citeSLR{brito2021identification,
kalia2021mono2micro, freitas2021refactoring}, but in today's world, where the
amount of data and information is constantly increasing, it would be ideal to
have a centralised location where architects, engineers, and developers can
access and utilise all the tools that are currently available as well as those
that will be developed in the future. Unfortunately, at the moment, no tool
that offers multiple options for decompositions with different possibilities
exists.

In this dissertation, we structurally analyse the state of the art in regards
to the migration of monolithic applications to the microservices architectural
style, mainly which tools help architects, engineers and developers in this
migration, and how automated they are.

To achieve this, the guidelines presented by Kitchenham and Charters
\cite{kitchenham2007guidelines} were followed while performing a systematic
literature review. The research protocol was defined at first and then followed
to ensure all results could be reproduced.

According to Kitchenham and Charters \cite{kitchenham2007guidelines}, research
questions should be specified as they will direct the entire review
methodology. The research questions formulated are as follows:

\emph{
  \begin{enumerate}[{RQ}1.]
    \item What tools already exist that aid in the migration process of
      monoliths to microservices?
      \begin{enumerate}[{RQ1.}1.]
        \item How do they take the monolith as input?
        \item How do they produce the microservice as output?
        \item Are they bound to a specific language?
      \end{enumerate}
    \item Is there an application that aggregates those tools to help
      architects, engineers and developers in their monolith decomposition?
  \end{enumerate}
}

Furthermore, we also develop an application that aims to aggregate existing
tools into a single platform and provide the means to extend and incorporate
new tools. This application offers a convenient and comprehensive way to access
and use various tools that help the decomposition from monoliths to
microservices and provide them with a perspective on several decomposition
proposals, allowing for easily comparable and different combinations options.

The purpose of the application is to answer the following research question:
\emph{
  \begin{enumerate}[{RQ}1.] \setcounter{enumi}{2}
    \item Can we devise an application that aggregates existing tools in a
      single comprehensive environment to help in the decomposition to
      microservices?
      \begin{enumerate}[{RQ3.}1.]
        \item Can the application exhibit a good usability for decomposing?
        \item Does the application provide a low workload?
      \end{enumerate}
  \end{enumerate}
}

It is important to mention that the final objective is not to create a new
technique for discovering microservices from a monolith system, but rather to
aggregate the already existing ones into a single and comprehensive framework.

We conducted an empirical study to evaluate the quality of the application,
focusing on assessing the usability and workload associated with its usage. The
study yielded favourable outcomes in terms of usability, indicating that
participants found the application to be user-friendly and intuitive. However,
the workload results were varied, suggesting that the application's impact on
reducing the overall workload during the decomposition process was not
consistently observed among participants.

The rest of this dissertation is structured as follows: \Cref{sec:background}
introduces the reader to various concepts of monolithic and microservices
architectures. \Cref{sec:slr} contains the literature review of the current
state of the art. The architecture and design of the application is explained
\Cref{sec:tool_design}. \Cref{sec:solution_development} discusses the
development of the application. In \Cref{sec:empirical_validation}, we present
the study design, its execution and corresponding analysis of results.
\Cref{sec:discussion} discusses the results obtained and their threats to
validity. Finally, \Cref{sec:conclusion} ends with with some conclusions and
future work.
