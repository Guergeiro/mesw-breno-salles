The purpose of an empirical study is to ensure that the study accurately
reflects the functionality and performance of the application. It is crucial to
minimise any potential issues arising from the study design or tasks that could
affect participants' responses, ensuring that their feedback reflects the
application's performance.

We employed an iterative approach of conducting mock studies with live support.
Participants' experiences and difficulties were carefully observed and noted
during these mock studies. Exactly two mock studies were conducted.

Based on the feedback received and the observations made during the mock
studies, we made improvements and adjustments to the study design and tasks.
The aim was to address and rectify any issues identified during the mock runs,
ensuring that the final study provided participants with a smoother and more
authentic experience.

To mitigate bias and prevent participants from being influenced by prior
knowledge or experience with the study, individuals who participated in the
mock runs were automatically excluded from the final study, helping to maintain
the integrity of the study results by ensuring that the responses obtained are
unbiased and representative of the application's performance.
