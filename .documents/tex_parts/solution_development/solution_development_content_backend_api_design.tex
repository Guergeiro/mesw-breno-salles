The backend architecture follows the principles of the REST style and utilises
JSON as the format for data transportation. The implemented REST API adheres to
the guidelines of Level 2 in the Richardson Maturity Model
\citeWEB{fowler-richardson-maturity-model}, which signifies a high level of API
maturity and conformity to REST principles.

\Cref{tab:rest_endpoints} provides a comprehensive depiction of the available
endpoints within the backend. These endpoints represent the various
functionalities and operations that can be performed through the API, allowing
clients to interact with the backend system and access its resources. By
conforming to Level 2 of the Richardson Maturity Model, the backend ensures
standardised and efficient communication between clients and the server,
enhancing interoperability and scalability.

\begin{table}[!htb] \caption{REST Endpoints} \label{tab:rest_endpoints}
  \begin{center}
    \begin{tabular}[c]{p{12em}|p{12em}|p{12em}}
      \textbf{Endpoint/Method} &
      \textbf{GET} &
      \textbf{POST} \\
      \hline {/users} & N/A & Creates a user \\
      \hline {/tools} & Retrieves all tools & N/A \\
      \hline {/results} & Retrieves all results & Creates a result \\
      \hline {/results/:id} & Retrieves a result that matches ``:id'' & N/A \\
      \hline {/decompositions} & Retrieves all decompositions & N/A \\
      \hline {/decompositions/:id} & Retrieves a decomposition that matches ``:id'' & N/A \\
      \hline {/decompositions/:id/export} & Exports a decomposition that matches ``:id'' & N/A \\
    \end{tabular}
  \end{center}
\end{table}

A lot of the time, when making calls to a REST API, there will be many results
to return. Therefore, we paginate the results to ensure responses are easier to
handle. Let us say the initial request asks for all the tools available; the
result could be a massive response with hundreds of thousands of tools. There
are better places to start than that. Therefore, we have built a limit on
results to ensure that only that amount of results will be returned. This limit
defaults to five results per page but can be changed at will.

We employed pagination on endpoints that may return multiple results, that is
on GET requests to \textit{/tools}, \textit{/results} and
\textit{/decompositions}. This allow us to confidently serve the content the
client needs, for example, serving the existing a tools and corresponding data
that composes each tool, thereby completing FR12.
