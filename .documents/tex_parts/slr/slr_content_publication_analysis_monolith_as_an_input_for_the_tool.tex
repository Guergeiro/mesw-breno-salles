The first aspect to be analysed is the input requirements for the tool. Despite
the growing interest in microservice migration using automated tools, the field
is still in its infancy, and the existing solutions tend to address specific
issues rather than being versatile. As a result, the inputs for these tools are
often rigid and not easily adjustable. For example, raw source code and OpenAPI
specification were possible ways tools use for identifying microservices from
monoliths and will be further discussed.

\subsubsection*{Source Code}

One potential method for providing input to a tool is by utilising source code
directly. Our research revealed that eighteen of the contributions analysed use
source code as input for their tools with multiple using Spring Boot or other
equivalent frameworks to help in understanding the overall code structure. One
reason for the use of frameworks is that they provide building blocks for
developers, meaning the core functionality of the framework is already in place
and developers simply fill in the gaps allowing for the framework to apply
inversion of control \cite{gamma1995design}. Since the behaviour of the
framework is kept intact, the tool can then safely analyse the overall code and
even apply the refactoring.

For instance, Freitas et al. \citeSLR{freitas2021refactoring} tool,
MicroRefact\footnote{\url{https://github.com/FranciscoFreitas45/MicroRefact}},
utilises Java source code to extract structural information by relying on the
Abstract Syntax Tree. This information is used to generate a list of candidate
microservices. They then leverage Spring Boot decorators, particularly those
utilising the Java Persistence API (JPA), to infer the entities of the database
and their relationships. This process then results in the output of working
Java code for each identified microservice.

\subsubsection*{OpenAPI}

In a microservices architecture, one of the common solutions for communication
between different microservices is through HTTP calls. Therefore, it is
reasonable to assume that identifying microservices within a monolithic
application could be done by examining their REST endpoints since they will be
exposed through HTTP protocols. This inspection of REST endpoints can also
serve as a guide for decomposing the monolithic application into smaller,
independent services. In fact, when the programming language was not a
determining factor, OpenAPI was commonly used as the standard for
distinguishing microservices from monolithic systems \citeSLR{al2019new,
sun2022expert}.

The tool proposed by Al-Debagy and Martinek \citeSLR{al2019new} utilises the
OpenAPI specification file to identify microservices within a monolithic
application. The tool begins by extracting the operation names from the OpenAPI
file, which are then input into the Affinity Propagation Algorithm
\cite{frey2007clustering}. This algorithm calculates the number of
microservices by analyzing the messages exchanged between data points.
Afterwards, clustering is performed by utilising the Silhouette coefficient
\cite{rousseeuw1987silhouettes} which results in the identification and
grouping of similar microservices, helping in the decomposition of the
monolithic system.

MsDecomposer\footnote{\url{https://github.com/HduDBSI/MsDecomposer}}
\citeSLR{sun2022expert} uses a similar approach to identify microservices
within a monolithic application. The first step is to calculate the similarity
of candidate topics and response messages among the APIs. Then, it constructs a
graph that represents the similarity between different APIs, where the APIs are
represented as nodes and the similarity score is the weight. Finally, it
applies a graph-based clustering algorithm on the constructed graph, which
helps to identify the candidate microservices.

This highlights that OpenAPI is a language-agnostic method for identifying the
architecture of software systems.

\subsubsection*{Other}

Besides the input types that have been previously mentioned, there are other
types that may not be able to create a new category but are still relevant to
the microservices identification process. For example, using a specific system
model as input \citeSLR{staffa2021pangaea} or utilising the history of changes
to understand in addition to common artefacts like classes and methods
\citeSLR{santos2021microservice}.
