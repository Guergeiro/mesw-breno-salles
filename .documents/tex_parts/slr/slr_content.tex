A systematic literature review is a type of review that aims to identify,
evaluate, and summarize the results of all studies that address a specific
research question or topic
\cite{kitchenham2007guidelines,kitchenham2009systematic,gough2017introduction}.
It involves following a specific methodology to identify, analyse, and
interpret all relevant evidence related to the research question being
addressed. The purpose of a systematic literature review is to provide a
comprehensive and up-to-date overview of the current state of knowledge on a
specific research question or topic. It is a critical appraisal of the existing
research and it can help identify gaps in the literature and inform future
research directions \cite{kitchenham2007guidelines}.

As per Kitchenham and Charters guidelines \cite{kitchenham2007guidelines}, a
systematic literature review (SLR) involves three phases: planning, conducting,
and reporting. The planning phase involves establishing the review protocol
based on the research questions and the need for the review. The conducting
phase involves selecting primary studies and applying the criteria established
in the review protocol to analyse them. Finally, the reporting phase involves
the creation of the report. These guidelines were loosely followed in the
development of this review.
