To ensure a thorough and comprehensive search for relevant publications in this
field, we will utilise a breadth-first search approach. This method involves
starting with a specific query string and selecting relevant publications from
a given database. We will then use a technique called snowballing to expand the
search and locate additional relevant publications. Snowballing involves
searching for citations and publications that are related to the initially
selected publications.

There are two types of snowballing that we will employ in this search: forward
snowballing and backward snowballing. Forward snowballing involves searching
for citations and publications using Google Scholar for the initially selected
publications. This process can be repeated multiple times, with each iteration
referred to as a level of snowballing. For this search, we will
perform two levels of forward snowballing, in which we extract the references
of the initially selected publications (level one) and then select the
references of those references (level two).

Backward snowballing involves searching for publications that have been cited
by the initially selected publications. This technique can also be repeated
multiple times, but for this search, we will only perform one
level of backward snowballing. This will include all previous publications
found during the forward snowballing step.

By utilising both forward and backward snowballing techniques, we aim to cast a
wide net and identify as many relevant publications as possible.

After completing the search for relevant publications in a given database using
the specified query string, we will move on to the next database. This approach
is advantageous because it allows us to efficiently locate relevant
publications while minimizing the number of duplicates that are analysed. By
searching multiple databases and using snowballing techniques, we can identify
a large number of relevant publications and eliminate the need to analyse many
of them in subsequent iterations.
