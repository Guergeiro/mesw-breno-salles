\documentclass{article}

\usepackage[utf8]{inputenc}
\usepackage{blindtext}
\usepackage{hyperref}
\usepackage{array}
\usepackage{float}
\usepackage{indentfirst}
\usepackage{enumerate}

\newcommand{\wrap}[1]{\parbox{.33\linewidth}{\vspace{1.5mm}#1\vspace{1mm}}}

\title{Refactoring-assisted migration of monoliths to microservices}
\author{Breno Salles} \date{January 2023}

\begin{document}

\maketitle

\begin{abstract}

  Lorem rem amet nesciunt voluptate ullam sapiente. Optio nemo alias nesciunt
  illo minima. Modi deleniti repellendus sint tempora quidem Eum quibusdam
  recusandae nostrum exercitationem dolores. Cum harum sapiente reiciendis
  inventore delectus neque! Deserunt unde hic sunt vel culpa Totam in corporis
  qui corrupti at magnam Dignissimos tempore quibusdam ipsa nobis illum Quis
  adipisci ex incidunt fuga nam sequi architecto adipisci facere Aut quam nulla
  cum sed expedita accusamus! Qui inventore molestias inventore quo officiis
  quibusdam, repellendus Facere rem a magnam aliquid doloribus distinctio
  distinctio? Labore dolorum quos veritatis neque culpa Saepe vero placeat qui
  odit excepturi Fugit laborum a eos

\end{abstract}

\section{Introduction} foo

Falar das questões?

\section{Background and Related Works} foo

\section{Systematic Literature Review} foo

\subsection{Research Methodology} \label{sub:research-methodology}

To address the research questions posed in the Introduction, the appropriate
research methods were utilized.

\subsubsection{Data sources and search strategy}

In terms of where to look for information, ACM Digital Library, Science Direct,
IEEE Xplore, Wiley, Springer Link, Engineering Village and Google Scholar were
the target databases present in Table \ref{tab:databases}.

\begin{table}[H]
  \caption{Databases}
  \label{tab:databases}
  \begin{center}
    \begin{tabular}[c]{l|l}
      \textbf{Search Engine} &
      \textbf{Website} \\
      \hline
      ACM Digital Library &
      \url{https://dl.acm.org/} \\
      \hline
      IEEE Xplore &
      \url{https://ieeexplore.ieee.org/} \\
      \hline
      Springer Link &
      \url{https://link.springer.com/} \\
      \hline
      Wiley &
      \url{https://onlinelibrary.wiley.com/} \\
      \hline
      Engineering Village &
      \url{https://www.engineeringvillage.com/} \\
      \hline
      Google Scholar &
      \url{https://scholar.google.com/} \\
    \end{tabular}
  \end{center}
\end{table}


% TODO: Aqui estava a pensar introduzir a palha a explicar o que são por alto

A breadth first search like approach will used, meaning that after selecting
relevant publications from a given database and query string, a one level
forward and backward snowballing of citations will be done, using Google
Scholar to search for citations and publications. When this is exhausted, the
next database and query string will be selected. This approach is helpful
because it reduces the number of publications to analyse during subsequent
databases queries as multiple publications will probably be previous analysed
in previous iterations.

In terms of research keywords, since the study of this thesis is mainly focused
``microservices", multiple ways of writing microservices should be used. As for
the practices that may help identification of microservices, keywords that help
this architectural refactoring should be included, things like ``migration",
``refactor", ``identification". It could also be useful to use ``monolith" (and
all its possible synonyms) to be the comparison against ``microservices",
although this can result in some extra publications not related to
microservices but instead related to ``service oriented architecture". An
expected outcome or conclusion of the publication could be included,
``approach" or even ``tool".

\begin{table}[H]
  \caption{Keywords}
  \label{tab:keywords}
  \begin{center}
    \begin{tabular}[c]{l|l}
      {\textbf{Focus}} &
      microservices \\
      \hline
      \wrap{{\textbf{Architectural\\Refactoring}}} &
      \wrap{
        migration, decomposition,\\
        identify, refactor, evolve,\\
        discover, transition
      } \\
      \hline
      {\textbf{Target}} &
      monolith \\
      \hline
      {\textbf{Outcome}} &
      approach, tool \\
    \end{tabular}
  \end{center}
\end{table}


\subsubsection{Selection Criteria}

In order to filter the publications, the title and the abstract will be
analysed and should mention some at least one of:

\begin{enumerate}[{IC}1.]
  \item A tool that automates process of migration of monoliths to
    microservices.
  \item Identification of microservices from monolith systems.
  \item Analysis of tools or approaches for migrating from monoliths to
    microservices.
\end{enumerate}

In cases of ambigous conclusions, further inspection of the publication may
be done. When this happens, and if relevant publications apply, conclusions
should also be taken into account.

As for more pratical approach for exclusion of publications, the criteria
will be:

\begin{enumerate}[{EC}1.]
  \item Publications that are not written in English or Portuguese.
  \item For Portuguese publications, English must be the language used in the
  \item Publication is not accessible.
\end{enumerate}

\subsection{Research Results}

In the first iteration of the review, the main purpose was figuring out how
many tools exist that were able to solve this research question or, at the
very least, help partially with it. In order to do this, one could not be
limited to tools that are documented in academic databases therefore, in
addition to the databases mentioned in \ref{sub:research-methodology},
\href{https://github.com}{GitHub}, \href{https://gitlab.com}{GitLab} and even
\href{https://duckduckgo.org}{DuckDuckGo} were searched for.

% \begin{table}
% \begin{tabular}{| m{4cm} | m{4cm}| m{2cm} m{2cm} |}

% \hline
% {Database} &
% {Query} &
% {Total number of results} &
% {Extracted results} \\
% \hline
% {https://dl.acm.org/} &
% {("microservice" OR "micro-service") AND ("migration" OR "identification") AND ("monolithic" OR "monolith") AND ("tool")} &
% {118} &
% {3} \\
% \hline
% {https://ieeexplore.ieee.org/} &
% {("microservice" OR "micro-service") AND ("migration" OR "identification") AND ("monolithic" OR "monolith") AND ("tool")} &
% {4} &
% {0} \\
% \hline
% {https://link.springer.com/} &
% {("microservice" OR "micro-service") AND ("migration" OR "identification") AND ("monolithic" OR "monolith") AND ("tool")} &
% {658} &
% {3} \\
% \hline
% {https://github.com/} &
% {https://github.com/search
% ?q=monolith+to+microservice} &
% {745} &
% {4} \\
% \hline
% {https://gitlab.com/} &
% {https://gitlab.com/search
% ?search=monolith\%20to\%20microservice} &
% {0} &
% {0} \\
% \hline
% {https://duckduckgo.org/} &
% {https://duckduckgo.com/
% ?q=monolith+to+microservices+tool} &
% {Infinity?} &
% {2} \\
% \hline

% \end{tabular}
% \end{table}

By doing this search, we can also backtrack from the references used and
check if the tools here were based on other work, but only a part of it and
therefore just implemented a specific approach.


\section{Conclusion and Future Work}

\end{document}
