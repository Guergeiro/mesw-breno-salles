\documentclass[conference]{IEEEtran}

\usepackage[utf8]{inputenc} \usepackage{blindtext} \usepackage{hyperref}
\usepackage{array} \usepackage{float} \usepackage{indentfirst}
\usepackage{enumerate}

\newcommand{\wrap}[1]{\parbox{.33\linewidth}{\vspace{1.5mm}#1\vspace{1mm}}}

\title{Refactoring-assisted migration of monoliths to microservices}
\author{Breno Salles} \date{January 2023}

\begin{document}

\maketitle

\begin{abstract}

  Lorem rem amet nesciunt voluptate ullam sapiente. Optio nemo alias nesciunt
  illo minima. Modi deleniti repellendus sint tempora quidem Eum quibusdam
  recusandae nostrum exercitationem dolores. Cum harum sapiente reiciendis
  inventore delectus neque! Deserunt unde hic sunt vel culpa Totam in corporis
  qui corrupti at magnam Dignissimos tempore quibusdam ipsa nobis illum Quis
  adipisci ex incidunt fuga nam sequi architecto adipisci facere Aut quam nulla
  cum sed expedita accusamus! Qui inventore molestias inventore quo officiis
  quibusdam, repellendus Facere rem a magnam aliquid doloribus distinctio
  distinctio? Labore dolorum quos veritatis neque culpa Saepe vero placeat qui
  odit excepturi Fugit laborum a eos

\end{abstract}

\section{Introduction} foo

Falar das questões?

\section{Background and Related Works} foo

\section{Systematic Literature Review} foo

\subsection{Research Methodology} \label{sub:research-methodology}

To address the research questions posed in the Introduction, the appropriate
research methods were utilized.

\subsubsection{Data sources and search strategy} \label{sub:search-strategy}

In terms of where to look for information, ACM Digital Library, Science Direct,
IEEE Xplore, Wiley, Springer Link, Engineering Village and Google Scholar were
the target databases present in Table \ref{tab:databases}.

\begin{table}[H] \caption{Databases} \label{tab:databases}
  \begin{center}
    \begin{tabular}[c]{l|l|l} \textbf{ID} & \textbf{Search Engine} &
      \textbf{Website} \\
      \hline DB.1 & ACM Digital Library & \url{https://dl.acm.org/} \\
      \hline DB.2 & IEEE Xplore & \url{https://ieeexplore.ieee.org/} \\
      \hline DB.3 & Springer Link & \url{https://link.springer.com/} \\
      \hline DB.4 & Wiley & \url{https://onlinelibrary.wiley.com/} \\
      \hline DB.5 & Wiley & \url{https://www.sciencedirect.com/} \\
      \hline DB.6 & Engineering Village &
      \url{https://www.engineeringvillage.com/} \\
      \hline DB.7 & Google Scholar & \url{https://scholar.google.com/} \\
    \end{tabular}
  \end{center}
\end{table}


% TODO: Aqui estava a pensar introduzir a palha a explicar o que são por alto

A breadth first search like approach will used, meaning that after selecting
relevant publications from a given database and query string, a two level
forward and backward snowballing of references will be done, using Google
Scholar to search for citations and publications. The two level simply means,
after selecting a publication to snowball, extracting its references is at
level one and then selecting the references of those references means level
two. The backward snowballing will then be applied just once for all previous
publications found, including ones selected in the forward snowballing step.

When this is exhausted, the next database and query string will be selected.
This approach is helpful because it reduces the number of publications to
analyse during subsequent databases queries as multiple publications will
probably be previous analysed in previous iterations.

In terms of research keywords, since the study of this thesis is mainly
focused ``microservices", multiple ways of writing microservices should be
used. As for the practices that may help identification of microservices,
keywords that help this architectural refactoring should be included, things
like ``migration", ``refactor", ``identification". It could also be useful to
use ``monolith" (and all its possible synonyms) to be the comparison against
``microservices", although this can result in some extra publications not
related to microservices but instead related to ``service oriented
architecture". An expected outcome or conclusion of the publication could be
included, ``approach" or even ``tool".

\begin{table}[H] \caption{Keywords} \label{tab:keywords}
  \begin{center}
    \begin{tabular}[c]{l|l} {\textbf{Focus}} & microservices \\
      \hline \textbf{Refactoring} & {\wrap{ migration, decomposition,\\
          identify, refactor, evolve,\\
      discover, transition }} \\
      \hline \textbf{Target} & monolith \\
      \hline \textbf{Outcome} & approach, tool \\
    \end{tabular}
  \end{center}
\end{table}


\subsubsection{Selection Criteria} \label{sub:selection-criteria}

In order to filter the publications, the title and the abstract will be
analysed and should mention some at least one of:

\begin{enumerate}[{IC}1.]
  \item A tool that automates process of migration of monoliths to
    microservices.
  \item Identification of microservices from monolith systems.
  \item Analysis of tools or approaches for migrating from monoliths to
    microservices.
\end{enumerate}

In cases of ambigous conclusions, further inspection of the publication may
be done. When this happens, and if relevant publications apply, conclusions
should also be taken into account.

As for more pratical approach for exclusion of publications, the criteria
will be:

\begin{enumerate}[{EC}1.]
  \item Publications that are not written in English or Portuguese.
  \item For Portuguese publications, English must be the language used in
    the
  \item Publication is not accessible.
\end{enumerate}

\subsection{Research Results}

In the first iteration of the review, the main purpose was figuring out how
many tools exist that were able to solve this research question or, at the
very least, help partially with it. In order to do this, one could not be
limited to tools that are documented in academic databases therefore, in
addition to the databases mentioned in Table \ref{tab:databases},
\href{https://github.com}{GitHub}, \href{https://gitlab.com}{GitLab} and
even \href{https://duckduckgo.org}{DuckDuckGo} were searched for, even
though they donot represent a scientific search engine.

By using some keywords mentioned in Table \ref{tab:keywords} we created an
initial trial query:

\begin{center}
  \emph{("microservice" OR "micro-service") AND ("migration" OR
  "identification") AND ("monolithic" OR "monolith") AND ("tool")}
\end{center}


The query was then applied to DB.1, DB.2 and DB.3 and after applying the
selection criteria mentioned in subsection \ref{sub:selection-criteria},
results were gathered and are presented in Table \ref{tab:tool-search}.

\begin{table}[H] \caption{Tool Search} \label{tab:tool-search}
  \begin{center}
    \begin{tabular}[c]{p{5.5em}|p{5em}|p{5em}} \textbf{Database} &
      \textbf{Total\newline number\newline of results} &
      \textbf{Extracted\newline Results} \\
      \hline DB.1 & {118} & {3} \\
      \hline DB.2 & {4} & {0} \\
      \hline DB.3 & {658} & {3} \\
    \end{tabular}
  \end{center}
\end{table}

As for the not so scientific search, the query would be essentilly typed into
their respective search engine and the results gathered are present in Table
\ref{tab:search-engine-tool-search}.

\begin{table}[H] \caption{Search Engine Tool Search}
  \label{tab:search-engine-tool-search}
  \begin{center}
    \begin{tabular}[c]{p{5.5em}|p{10em}|p{4em}|p{5em}}
      \textbf{Search\newline Engine} &
      \textbf{Query} &
      \textbf{Total\newline number\newline of results} &
      \textbf{Extracted\newline Results} \\
    \hline GitHub &
      {https://github.com/search\newline?q=monolith+to\newline+microservice} & {745} & {4}
      \\
      \hline GitLab &
      {https://gitlab.com/search\newline?search=monolith\%20\newline to\%20microservice} & {0} &
      {0} \\
      \hline DuckDuckGo &
      {https://duckduckgo.com/\newline?q=monolith+to\newline+microservices+tool} & {Infinity} &
      {2} \\
    \end{tabular}
  \end{center}
\end{table}
By doing this search, we can also backtrack from the references used and
check if the tools here were based on other work, but only a part of it and
therefore just implemented a specific approach.

As for the second iteration of the review, the focus was more about finding
other approaches for migrating from monolith to microservices that might have
been documented and not implemented. With this, it will be easier to understand
where the state of the art starts and ends, what is missing from at the moment
and what can be improved on.

Since it was a start of a new iteration, the results of the previous ``test
run" were not taken into account when extracting literature and only after this
iteration, the already analysed literature will not be reanalysed.

Relying on the keywords identified in Table \ref{tab:keywords}, the following query was tried:

\begin{center}
  \emph{(microservice* OR micro?service*) AND (migrat* OR identif*) AND (monolith*) AND (migrat* NEAR/2 (process* OR approach*))}
\end{center}

\begin{table}[H] \caption{DB.1 Results} \label{tab:db1-search}
  \begin{center}
    \begin{tabular}[c]{p{5em}|p{5em}} \textbf{Total\newline number\newline of
      results} & \textbf{Extracted\newline New Results} \\
      \hline{450} & {12} \\
    \end{tabular}
  \end{center}
\end{table}

The next step was going through the papers references to try to find other
related publications that is, by applying forward and backward snowballing as
stated in \ref{sub:search-strategy}

\begin{table}[H] \caption{DB.1 Snowballing Results} \label{tab:db1-snowballing}
  \begin{center}
    \begin{tabular}[c]{p{8em}|p{8em}|p{8em}}
      \textbf{1st Forward} &
      \textbf{2nd Forward} &
      \textbf{Backward} \\
      \hline{23} &
      {2} &
      {45} \\
    \end{tabular}
  \end{center}
\end{table}



\subsection{Publications Grouping}

With the huge amount of publications selected as candidates for final reading,
there was a necessity diluting the list of publications even more. To achieve
this, categorizing them for later selection and better priotization was the
approach we went for. In this case, we arrived at four main groups to be
categorize each publication:

\begin{itemize}
  \item The \textbf{approach} used for identifying microservices from monoliths.
  \begin{itemize}
    \item \textit{Data flow}.
    \item \textit{Dependency analysis}.
    \item \textit{Execution log}.
    \item \textit{etc}.
  \end{itemize}
  \item The current \textbf{status} of the publication.
  \begin{itemize}
    \item It only explains the \textit{method} at a high level.
    \item Has implementation details with the \textit{algorithm} on how to identify.
    \item Already has a working \textit{tool}.
  \end{itemize}
  \item The \textbf{language} in which that it targets.
  \begin{itemize}
    \item \textit{Java}.
    \item \textit{Cpp}.
    \item \textit{C}.
    \item \textit{Language Agnostic}.
    \item \textit{etc}.
  \end{itemize}
\end{itemize}

% \begin{table}[H] \caption{Approach grouping}
%   \begin{center}
%     \begin{tabular}[c]{p{8em}|p{4em}}
%       \textbf{Approach} &
%       \textbf{Amount} \\
%       \hline Static analysis & {13} \\
%       \hline Model based & {9} \\
%       \hline Dependency analysis  & {7} \\
%       \hline Domain analysis  & {6} \\
%       \hline Data flow  & {6} \\
%       \hline Data & {5} \\
%       \hline Cohesion & {4} \\
%       \hline Multiple & {4} \\
%     \end{tabular}
%   \end{center}
% \end{table}

\subsection{Selected Work}



\section{Conclusion and Future Work}

\end{document}
