The threat to the conclusion validity is related to the participants involved.

As outlined in \Cref{sub:analysis}, the study was conducted with a relatively
small sample size of fifteen participants, which may lead to reduced
statistical power. To help mitigate this, we invited participants with diverse
backgrounds, including varying levels of industry expertise, academic
expertise, and familiarity with microservices.

Furthermore, the asynchronous nature of the experiment introduces a potential
challenge in controlling the participants' environment and level of focus
during the tasks. To mitigate this issue, the study aimed to provide a smooth
and user-friendly experience, as discussed in Section \Cref{sub:pre_study}.

There is also the argument that drawing general conclusions from only two tasks
presents a limitation in terms of generalizability. It is hard to balance the
subject's interest in the study while trying to extract as much data from them
as possible. However, to address this limitation, two projects with different
sizes and domains were included in the study (as explained in
\Cref{sub:instrumentation}).
