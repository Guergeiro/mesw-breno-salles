Microservices are becoming more popular in the development of new applications.
Many businesses, both large and small, are using microservices to design and
deliver applications more rapidly and efficiently
\cite{richardson-microservices}. Microservices are especially well-suited for
distributed, cloud-based systems, where they may benefit from the cloud's
flexibility and scalability \cite{fowler-microservices-prerequisites}.

We performed a literature review on the subject of migratin architecture from
monolithic applications to microservices for this study. A total of one hundred
and six primary research contributions were chosen, categorised, and analysed
using a clear research protocol in order to collect pertinent migration data. A
tool's target programming languages, the processes it uses to convert
monolithic inputs into microservices, and the output of these identified
microservices are the subject of analysis in this study.

Tool input needs were examined first. Despite increased interest in
microservice migration using automated tools, the topic is still young, and
current solutions have strict inputs rather than being adaptable. Raw source
code and OpenAPI were one method tools to identify microservices from
monoliths.

As for how existing tools output their identified microservices, the outcome
would ideally be fully functional code ready for deployment, as it would make
the migration process smoother and ensure that the resulting microservices have
all necessary components, thus reducing the effort needed for manual migration.
From the research analysed, common outputs from microservices identification
tools include a list of candidates and source code.

Based on the literature review and analysis, we found limited tools to aid the
migration process from monoliths to microservices. Seven free and open-source
tools were identified, each with varying levels of completeness.

In terms of an application that aggregates these tools to assist architects,
engineers, and developers in microservice migration, we have yet to find an
existing application to fulfil this role. Therefore, the dissertation proposes
a solution to this gap by providing a user-friendly and consolidated platform
for microservice migration activities.

The application architecture consists of three distinct components. The
frontend is responsible for the interface between the tool logic and the user.
The backend serves as a bridge between the frontend, the database of available
tools, and the tool domain. Each tool's adapter provides a consistent interface
for interacting with the tool runtime. Each component of the application
architecture is be deployed as a microservice in a Docker container to
facilitate scalability and cross-platform deployment. This approach allows us
to process multiple jobs concurrently and avoid potential bottlenecks in the
communication between the tool runtime and the backend.

Decomposing monolithic architectures into microservices primarily involves
labour-intensive and time-consuming manual work, coupled with the difficulty of
identifying functional units due to the complex analysis required across
multiple dimensions of the software architecture. However, the application
proposed in this study's experiment favourably impacted the participants.

As evidenced by the positive evaluations and high scores obtained in the SUS,
participants perceived the application as user-friendly, intuitive, and easy to
navigate.

Regarding the workload, the application did not yield significant effects on
the decomposition process of monolithic architectures, as the performance
results were mixed.
