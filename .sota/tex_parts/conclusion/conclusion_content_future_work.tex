Given that software development is never finished, like fashion is never
finished, the same principle applies to this work. There are two areas for
future improvements: enhancements to the application and refinements in the
evaluation process.

In terms of application improvements, valuable insights were obtained through
the feedback questionnaire (depicted in \Cref{fig:questionaire_feedback}),
leading to the identification of several potential future enhancements:

\begin{itemize}
  \item \textbf{Enhancing metadata descriptions for improved comparisons:} The
    application only presents metadata metrics without accompanying
    explanations for each metadata. It would be beneficial to provide
    descriptions for metadata attributes to facilitate better understanding and
    comparisons.

  \item \textbf{Implementing decomposition tracking features:} Introducing
    functionality to track decompositions would enable users to manage and
    differentiate between various decompositions, like selecting favourites or
    marking decompositions as already viewed. Such features assist users in
    discarding irrelevant decompositions while retaining others for future
    comparisons.

  \item \textbf{Establishing bidirectional relationships between microservices
    and modules:} When focusing on a specific microservice, the tool displays
    the list of modules that compose it. However, this relationship is
    unidirectional. It would be advantageous to establish a two-way
    association, enabling users to select a module and visualise all the
    corresponding microservices across different decompositions.
\end{itemize}

In terms of evaluation improvements, some considerations should be taken into
account for future studies, specifically:

\begin{itemize}
  \item \textbf{Expanding the respondent pool:} Increasing the number of
    participants in evaluations can enhance the reliability and
    generalizability of the findings. A more extensive and diverse group of
    respondents can capture a more comprehensive range of perspectives and
    experiences, leading to more robust conclusions.
  \item \textbf{Incorporating a comparative approach:} Participants should be
    engaged in both tasks, one without utilising the tool and another while
    using the tool. By comparing the outcomes and experiences between these two
    scenarios, the added value and impact of the tool can be assessed more
    accurately. Additionally, employing different tasks with similar-sized
    projects can help evaluate the tool's performance across various contexts
    and validate its usefulness in different scenarios.
  \item \textbf{Broadening the task pool and project size range:} Diversifying
    the available tasks and expanding the range of project sizes used in
    evaluations contribute to a more comprehensive assessment of the tool.
\end{itemize}
