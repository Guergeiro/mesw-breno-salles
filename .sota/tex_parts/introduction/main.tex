\section{Introduction} \label{sec:introduction}

Microservices is an architectural style that evolved from Service Oriented
Architecture (SOA). Just like SOA, microservices are an alternative to
monolithic architecture. The main contrasts are that, while monolithic
applications are software systems with a single, integrated codebase that
includes all necessary components, and features
\cite{kazanavivcius2019migrating}, microservices tend to be separated, and
loosely coupled \cite{newman2021building}. Also while monoliths tend to be
easier to develop they may scale poorly and are harder to maintain when
compared to microservices \cite{newman2019monolith}. Microservices are
increasingly being used in the development of modern applications, particularly
in the areas of cloud computing and DevOps \cite{ren2018migrating}. Many
organizations, including large enterprises and startups, are adopting
microservices as a way to build and deploy applications more quickly and
efficiently \cite{richardson-microservices}. Microservices are particularly
well-suited for distributed, cloud-based environments, where they can take
advantage of the flexibility and scalability of the cloud
\cite{fowler-microservices-prerequisites}. This type of architecture is already
being applied in multiple well-known companies, like Uber, Netflix, eBay
\cite{microservices-users}, and also being followed by the rest of the herd
when compared to monolith architecture \cite{taibi2017processes}.

Refactoring from monoliths to microservices is a heavily debated topic both in
the academic world and the industry. The main takes from this debate are that
refactoring is difficult and time-consuming, especially for companies with
limited resources who struggle with migrating their already existing monolithic
applications to microservices. To help address this, some tools were developed
that help with the refactor \citeSLR{brito2021identification,
kalia2021mono2micro, freitas2021refactoring}, but in today's world, where the
amount of data and information is constantly increasing, it would be ideal to
have a centralised location where architects, engineers, and developers can
access and utilise all the tools that are currently available as well as those
that will be developed in the future. Unfortunately, at the moment, no tool
that offers multiple options for migrations with different possibilities
exists.

The goal of this study is to structurally analyse the state of the art in
regards to the migration of monolithic applications to microservices
architectural style, mainly how tools are helping architects, engineers and
developers in this migration, and how automated they are. Furthermore, in the
following thesis, the purpose is to develop a tool which aims to aggregate
existing tools into a single platform, as well as provide the means to extend
and incorporate new tools. This tool will offer a convenient and comprehensive
way to access and use a variety of tools that help the refactor from monoliths
to microservices as well as provide them with a perspective on several
migration proposals, allowing for easily comparable and different combinations
options.

To achieve this, the guidelines presented by Kitchenham and Charters
\cite{kitchenham2007guidelines} were followed while performing a systematic
literature review. The research protocol was defined at first and then followed
to ensure all results could be reproduced.

According to Kitchenham and Charters \cite{kitchenham2007guidelines}, research
questions should be specified as they will direct the entire review
methodology. The research questions formulated are as follows:

\emph{
  \begin{enumerate}[{RQ}1.]
    \item What tools already exist that aid in the migration process of
      monoliths to microservices?
    \begin{enumerate}[{RQ1.}1.]
      \item How do they take the monolith as input?
      \item How do they produce the microservice as output?
      \item Are they bound to a specific language?
    \end{enumerate}
    \item How can an aggregator of those tools help architects, engineers
      and developers in their microservice migration?
  \end{enumerate}
}
