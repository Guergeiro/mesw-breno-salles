The study was disseminated among professional colleagues, academic peers, and
within the open-source communities to ensure diverse participants. The
optimisation described in \Cref{sub:pre_study} allowed for asynchronous
participation, making it more convenient for participants to engage in the
study. Nevertheless, real-time assistance was provisioned in case of issues
arising.

Participants were required to validate their knowledge of microservices.
Initially, those with limited prior knowledge could watch a small video to
provide a basic understanding of microservices concepts. Following this,
participants completed a questionnaire that collected background information
such as age, gender, industry and academic experience, educational level, and
proficiency in programming languages.

A tutorial follows, demonstrating an example task and explaining the expected
approach. Participants were then assigned their specific tasks and asked to
replicate the steps demonstrated in the tutorial.

Afterwards, participants were asked to identify the microservice they believed
to be the best for their assigned task. Additionally, participants filled out
the System Usability Scale and Task Load Index questionnaires to assess the
usability and perceived workload of the application.

Finally, an optional section was included where participants were encouraged to
provide additional feedback on the tool, allowing them to share any additional
thoughts, suggestions, or comments they had.
