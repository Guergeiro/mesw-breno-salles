The study was distributed to work colleagues, college colleagues and open-source communities the author belongs to. Because of the study optimisation described in \Cref{sub:pre_study}, it was not necessary to have a syncronous participation from the respondents, which made it easier to invite them to participate, although live support was offered if ever an issue occured.
To start, the participants had to validate their knowledge of microservices and, to help with that, a small video was added in order to help those who had not prior knowledge of microservices. Afterwards, they would answer a small questionnaire regarding their background, where they would state their current age, their gender, their experience in the industry and academia, their instruction level as well as their prowness with certain programming languages. The tutorial follows where an example task is used to demonstrate what it is expected for the task and then, the participants replicate the steps of the tutorial using the task assigned to them. Finally, the users state which microservice they understand to be the best one as well as felling in the System Usability Scale and Task Load Index. There is also an optional section where participants are invited to give some extra feedback for the tool.
