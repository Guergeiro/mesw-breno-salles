the national aeronautics and space administration (nasa) developed nasa task
load index (nasa-tlx) to evaluate subjective workload. it employs a
multi-dimensional rating approach considering the weighted average ratings of
six subscales. these subscales assess various aspects of workload experienced
by participants, including subjective mental demand, physical demand, time
demand, performance, effort, and frustration levels in a ten-point response
scale \cite{hart1988development}.

in the study context, the raw task load index (raw-tlx) is used as a simplified
version of the nasa-tlx. one potential difference is that nasa-tlx incorporates
a weighting process to adjust each scale to an individual's perception of
workload \cite{hart1988development}. in contrast, raw-tlx omits this weighting
process to maintain simplicity and ease of application \cite{hart2006nasa}.

In alignment with the use of Raw-TLX, the study also omitted one of the scales,
Physical Demand, as it did not apply to the specific context of the task. The
omission of this scale was due to its lack of relevance in assessing the
workload associated with the decomposition tasks in the study. Not omitting the
scale could skew the overall score since all scales have equal weight, that is,
\( \frac{6}{100} \). From all scales of Raw-TLX, only the Performance scale
considers higher evaluations as positive. As for the other four scales, having
a high score is considered harmful.

The Raw Task Load Index (Raw-TLX) was conducted to collect information on the
effort required to complete the monolith decomposition tasks using the
application. Then we calculated each category's average and median scores
separately for each task. Task 1 and 2 results and scores are presented in
\Cref{tab:separated_task_1,tab:separated_task_2} as well as visualy in
\Cref{fig:separated_task_1,fig:separated_task_2}, respectively.

\Cref{tab:tasks_results} combines Task 1 and Task 2 results and the general
average and median to provide a comprehensive overview of the perceived
workload across both tasks. The visualisation in \Cref{fig:tasks_results}
allows for comparing and analysing the workload distribution across the
different categories for the overall study. Each line in all three graphs
represents a different participant and the task they were each assigned.

\begin{table}[!htb] \caption{Task 1 Results Table} \label{tab:separated_task_1}
  \begin{center}
    \begin{tabular}[c]{p{4em}|p{4em}|p{4em}|p{4em}|p{4em}|p{4em}}
      \textbf{} &
      \textbf{Mental Demand} &
      \textbf{Temporal Demand} &
      \textbf{Perfor\-mance} &
      \textbf{Effort} &
      \textbf{Frustration} \\
      \hline Average & 4.71 & 4.14 & 4.71 & 4.86 & 3.43 \\
      \hline Median & 5.0 & 4.0 & 4.0 & 5.0 & 3.0 \\
    \end{tabular}
  \end{center}
\end{table}

\begin{figure*}[!htb]
  \centering
  \includegraphics[width=\textwidth]{task1_results}
  \caption{Task 1 Graph}
  \label{fig:separated_task_1}
\end{figure*}

\begin{table}[!htb] \caption{Task 2 Results Table} \label{tab:separated_task_2}
  \begin{center}
    \begin{tabular}[c]{p{4em}|p{4em}|p{4em}|p{4em}|p{4em}|p{4em}}
      \textbf{} &
      \textbf{Mental Demand} &
      \textbf{Temporal Demand} &
      \textbf{Performance} &
      \textbf{Effort} &
      \textbf{Frustration} \\
      \hline Average & 4.13 & 4.75 & 5.75 & 5.00 & 3.25 \\
      \hline Median & 4.0 & 5.0 & 7.0 & 5.5 & 2.5 \\
    \end{tabular}
  \end{center}
\end{table}

\begin{figure*}[!htb]
  \centering
  \includegraphics[width=\textwidth]{task2_results}
  \caption{Task 2}
  \label{fig:separated_task_2}
\end{figure*}

\begin{table}[!htb] \caption{Tasks Results Table} \label{tab:tasks_results}
  \begin{center}
    \begin{tabular}[c]{p{4em}|p{4em}|p{4em}|p{4em}|p{4em}|p{4em}}
      \textbf{} &
      \textbf{Mental Demand} &
      \textbf{Temporal Demand} &
      \textbf{Perfor\-mance} &
      \textbf{Effort} &
      \textbf{Frustration} \\
      \hline Average & 4.40 & 4.47 & 5.27 & 4.93 & 3.33 \\
      \hline Median & 4.0 & 4.0 & 5.0 & 5.0 & 3.0 \\
    \end{tabular}
  \end{center}
\end{table}

\begin{figure*}[!htb]
  \centering
  \includegraphics[width=\textwidth]{tasks_results}
  \caption{Tasks Results}
  \label{fig:tasks_results}
\end{figure*}
