For microservices identification, it is important to understand how existing
tools output their identified microservices in order to cater to the needs of
users. The ideal outcome would be a fully functional code ready to be deployed,
as it makes the migration process smoother, ensuring that the resulting
microservices have all the necessary components and reducing the effort needed
for manual migration. From the work that was analysed, tools that output a list
of candidates and tools that output source code are more relevant to take into
consideration and will be further discussed.

\subsubsection*{Candidates List}

A prevalent approach for identifying microservices in a monolithic system is by
producing a candidate list. This approach is used in most of the publications
and tools found, and it is a way of providing an organized and structured
output of the microservices that can be derived from the monolithic
application, in order to facilitate the migration process. The candidate list
is commonly used as a guide or checklist for architects, engineers, and
developers to assist in the actual partitioning of the application. It
typically includes but is not limited to, data entities, interfaces, methods,
and other relevant system components that are used as references to guide the
migration process.

The tool proposed by Al-Debagy and Martinek \citeSLR{al2019new} uses OpenAPI as
input to identify microservices within a monolithic application. One of the
limitations of their output is that it only provides a list of candidates,
which may not include information about the relationships and interactions
between each microservice. This can be an inconvenience when assigning
different development teams or groups with the task of applying the migration,
as they may not have enough information to understand how the microservices
interact and depend on each other, making it harder to assign and split the
workload accordingly.

There are other tools available that output a more informative result for
microservices identification. Even though they still output candidate lists,
where no migration is done yet, these tools provide more detailed information
about the relationships and connections between microservices, some of them
even including visual feedback such as clusters and call context tree diagrams
\citeSLR{krishna2021transforming, kalia2021mono2micro, kalia2020mono2micro,
nakazawa2018visualization}. This additional information can be very beneficial
for architects and developers, as it makes it easier for them to understand the
connections and dependencies between microservices, and make informed decisions
about how to proceed with the migration process.

\subsubsection*{Source Code}

Generating the final output of the migration process as source code that is
ready to be deployed would be the ideal outcome for most cases, as it would
lessen the efforts needed to migrate, but it is not yet commonly used in
current literature. Out of the tools that were analysed, only two of them
employ this method. One of them is the work of Freitas et al.
\citeSLR{freitas2021refactoring}, and another is Mono2Micro
\citeSLR{krishna2021transforming, kalia2021mono2micro, kalia2020mono2micro,
nakazawa2018visualization} that in an update explained in a YouTube video
\footnote{\url{https://youtu.be/NXno1fUoC8U}} they now output Java code.
