Based on the literature review and analysis presented in the preceding
sections, the research question \textit{RQ1 - ``What tools already exist that
aid in the migration process of monoliths to microservices?''} was examined. As
mentioned in \Cref{sub:publications_grouping_selection}, limited tools are
available to decompose monolithic architectures into microservices. We
identified seven free and open-source tools, each with varying degrees of
completeness, as discussed in the relevant literature. However, it is worth
noting that the most promising tool identified, IBM's Mono2Micro
\citeSLR{krishna2021transforming, kalia2021mono2micro, kalia2020mono2micro}, is
not publicly accessible to the general public.

Addressing the following research question, \textit{RQ2 - ``Is there an
  application that aggregates those tools to help architects, engineers, and
developers in their microservice migration?''} the research findings revealed a
lack of such an application. We found no existing application that served as an
aggregator of multiple decomposition tools, offering users a graphical and
unified interface. Therefore, this dissertation addresses this gap by proposing
a solution that precisely fulfils the need for an application that aggregates
decomposition tools, providing a user-friendly and consolidated platform for
microservice migration activities.
