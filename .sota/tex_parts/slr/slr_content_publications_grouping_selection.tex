Given the large number of publications that were identified as potential
candidates for further analysis, it was necessary to further reduce the list to
a more manageable size. To accomplish this, we employed a categorization
approach in order to better organize and prioritize the publications for later
selection. This allowed us to select and analyse the most relevant publications
for our purposes. Through this process, we arrived at three main categories
that were derived from RQ1 into which we could place each publication. This
will be especially relevant when creating the new tool, by enhancing the
possibility of integrating various tools that employ different approaches, in
order to provide the developer with multiple perspectives, which may facilitate
the ability to make comparisons and informed decisions.

\begin{itemize}
  \item The \textbf{approach} used for identifying microservices from
    monoliths, \Cref{tab:approach-grouping}.
  \begin{itemize}
    \item \textit{Data flow}.
    \item \textit{Dependency analysis}.
    \item \textit{Execution log}.
    \item \textit{etc}.
  \end{itemize}
  \item The current \textbf{status} of the publication,
    \Cref{tab:status-grouping}.
  \begin{itemize}
    \item It only explains the \textit{method} at a high level.
    \item Has implementation details with the \textit{algorithm} on how to
      identify.
    \item Already has a working \textit{tool}.
  \end{itemize}
  \item The \textbf{language} in which that it targets,
    \Cref{tab:language-grouping}.
  \begin{itemize}
    \item \textit{Java}.
    \item \textit{Cpp}.
    \item \textit{C}.
    \item \textit{Language Agnostic}.
    \item \textit{etc}.
  \end{itemize}
\end{itemize}

The publications that were selected are grouped in
\Cref{tab:approach-grouping,tab:status-grouping,,tab:language-grouping}. It is
important to note that the papers analysed in this study were classified into
multiple categories, as opposed to a singular classification. To facilitate a
more comprehensive understanding, the classified papers can be viewed on the
online spreadsheet\footnote{\url{https://bit.ly/publication-grouping}}.

\begin{table}[!htb] \caption{Approach grouping} \label{tab:approach-grouping}
  \begin{center}
    \begin{tabular}[c]{p{12em}|p{4em}}
      \textbf{Approach} &
      \textbf{Amount} \\
      \hline Data flow & {8} \\
      \hline Control flow & {7} \\
      \hline Dynamic analysis & {11} \\
      \hline Semantic analysis & {4} \\
      \hline Problem frames & {1} \\
      \hline Model based & {10} \\
      \hline Static analysis & {13} \\
      \hline Dependency analysis & {15} \\
      \hline Multi objective & {1} \\
      \hline Feature analysis & {7} \\
      \hline Data analysis & {6} \\
      \hline REST & {4} \\
      \hline Graph based & {2} \\
      \hline Domain analysis & {10} \\
      \hline Neural analysis & {2} \\
      \hline Layer & {1} \\
      \hline Business analysis & {3} \\
      \hline Strangler pattern & {1} \\
      \hline Code change history & {1} \\
      \hline Contributor based & {1} \\
      \hline Logs analysis & {1} \\
      \hline Transactional contexts & {1} \\
      \hline Execution flow & {1} \\
      \hline Unknown & {2} \\
    \end{tabular}
  \end{center}
\end{table}

\begin{table}[!htb] \caption{Status grouping} \label{tab:status-grouping}
  \begin{center}
    \begin{tabular}[c]{p{12em}|p{4em}}
      \textbf{Status} &
      \textbf{Amount} \\
      \hline Method & {48} \\
      \hline Algorithm & {7} \\
      \hline Tool & {25} \\
      \hline Unknown & {1} \\
    \end{tabular}
  \end{center}
\end{table}

\begin{table}[!htb] \caption{Language grouping} \label{tab:language-grouping}
  \begin{center}
    \begin{tabular}[c]{p{12em}|p{4em}}
      \textbf{Language} &
      \textbf{Amount} \\
      \hline Agnostic & {59} \\
      \hline Java & {16} \\
      \hline Ruby & {1} \\
      \hline Python & {3} \\
      \hline Unknown & {4} \\
    \end{tabular}
  \end{center}
\end{table}

Having evaluated most of the literature in regard to tools that help with the
migration of monoliths to microservices, we need to select those that are most
relevant for the purpose of this thesis. Given our focus on tools and their
implementation, we will prioritise works that have already developed a tool and
made it available for a free inspection and use. Therefore, if a publication
does not provide a link to the tool or instructions for self-hosting or
deploying it, it is not worth further consideration. This will help us to focus
our efforts on publications that provide practical and useful information about
tools and their implementation. The tools that fulfilled these requirements are:

\begin{itemize}
  \item \url{https://github.com/HduDBSI/MsDecomposer} \citeSLR{sun2022expert}
  \item \url{https://github.com/FranciscoFreitas45/MicroRefact} \citeSLR{freitas2021refactoring}
  \item \url{identification-of-microservices-from-monolithic-applications-through-topic-modelling.md} \citeSLR{brito2021identification}
  \item \url{https://github.com/gmazlami/microserviceExtraction-backend} \citeSLR{agarwal2021monolith}
  \item \url{https://github.com/socialsoftware/mono2micro} \citeSLR{nunes2019monolith}
  \item \url{https://github.com/antbucc/Migration} \citeSLR{bucchiarone2020model}
  \item \url{https://github.com/tiagoCMatias/monoBreaker} \citeSLR{matias2020determining}
\end{itemize}
