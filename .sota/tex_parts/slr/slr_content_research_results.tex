The initial query mentioned in the \Cref{sub:query-definition} was applied to
\href{https://github.com}{GitHub}, \href{https://gitlab.com}{GitLab} and
\href{https://duckduckgo.org}{DuckDuckGo}, the query would be essentially typed
into their respective search engine and the results gathered as well as the
query are presented in \Cref{tab:search-engine-tool-search}.

The reason we used these search engines in detriment of others were:

\begin{itemize}
  \item GitHub and GitLab: both are one of the most popular source code hosting
    platforms, which would increase our chance of finding relevant results.
  \item DuckDuckGo: the one we thought would less likely influence results.
\end{itemize}

\begin{table}[!htb] \caption{Search Engine Tool Search}
  \label{tab:search-engine-tool-search}
  \begin{center}
    \begin{tabular}[c]{p{5.5em}|p{10em}|p{5em}|p{4em}}
      \textbf{Search\newline Engine} &
      \textbf{Query} &
      \textbf{Total\newline number\newline of results} &
      \textbf{Extracted\newline Results} \\
      \hline
        GitHub &
        \url{https://github.com/search?q=monolith+to+microservice} &
        {745} &
        {4} \\
      \hline
        GitLab &
        \url{https://gitlab.com/search?search=monolith\%20to\%20microservice} &
        {0} &
        {0} \\
      \hline
        DuckDuckGo &
        \url{https://duckduckgo.com/?q=monolith+to+microservices+tool} &
        {Uncountable} &
        {2} \\
    \end{tabular}
  \end{center}
\end{table}

Through this search process, we can also trace the references used in these
publications to determine if the tools described were based on previous work,
but only implemented a specific approach. This will help us to understand the
context and origins of these tools and how they fit into the broader landscape
of research in this area.

Applying the the query to the databases yielded 1394, with 34 that were
extracted for having passed the selection criteria defined in
\Cref{sub:selection-criteria}, as shown in \Cref{tab:other-db-search}.

\begin{table}[!htb] \caption{DB Results} \label{tab:other-db-search}
  \begin{center}
    \begin{tabular}[c]{p{5em}|p{5em}|p{5em}}
      \textbf{Database} &
      \textbf{Total\newline number\newline of results} &
      \textbf{Extracted\newline Results} \\
      \hline {ACM} & {568} & {15} \\
      \hline {IEEE} & {4} & {0} \\
      \hline {SPL} & {678} & {3} \\
      \hline {WLY} & {9} & {3} \\
      \hline {SCI-D (1st)} & {21} & {1} \\
      \hline {SCI-D (2nd)} & {0} & {0} \\
      \hline {ENG-V} & {114} & {12} \\
    \end{tabular}
  \end{center}
\end{table}

In the case of Science Direct, as presented in \Cref{tab:other-db-search},
two queries were done. The main reason for this is that Science Direct is
limited to 7 \textit{OR} conditions, therefore it was necessary to split it
into two queries where it does not affect the general condition. In the
specific case, the \textit{``evolv''} keyword was moved into a separate query.
Also, Science Direct automatically accepts truncations without using the
\textit{``*''} char. The two queries are:

\begin{enumerate}
  \item \emph{(microservice OR ``micro-service'') AND ( migrat OR decompos OR
    identif OR refactor OR extract OR discover OR transition)}
  \item \emph{(microservice OR ``micro-service'') AND (evolv)}
\end{enumerate}

After reviewing the references of the identified papers and applying forward
and backward snowballing techniques, we were able to locate additional related
publications and expand the scope of our search as demonstrated in
\Cref{tab:db1-snowballing}. This helped us to increase the number of relevant
publications that we were able to consider in the next steps of the process.

\begin{table}[!htb] \caption{Snowballing Results} \label{tab:db1-snowballing}
  \begin{center}
    \begin{tabular}[c]{p{8em}|p{8em}|p{8em}}
      \textbf{1st Forward} &
      \textbf{2nd Forward} &
      \textbf{Backward} \\
      \hline{23} &
      {2} &
      {45} \\
    \end{tabular}
  \end{center}
\end{table}

Having iterated over the results and reviewing the references of the newly
found publications, we did not identify any additional publications that were
worth including in the final list. This marked the end of our general search
for relevant publications. We were able to find 106 relevant publications.

All the results that were analysed from of the search are available in a gist\footnote{\url{https://gist.github.com/Guergeiro/c3baefb0ac6fdf673866f6515f1416a3}}.
