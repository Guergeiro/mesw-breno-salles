To identify relevant publications for this research, we will utilise a range of
keywords related to the topic of microservices. These keywords will include
various phrases and terms used to describe microservices. As for the practices
that may help identification of microservices, keywords that help this
architectural refactoring should be included, such as ``migration",
``refactor", ``identification". It could also be useful to use ``monolith" (and
all its possible synonyms) to be the comparison against ``microservices",
although this can result in some extra publications not related to
microservices but instead related to ``service-oriented architecture". An
expected outcome or conclusion of the publication could be included,
``approach" or even ``tool". The main keywords that will be used are present in
\Cref{tab:keywords}.

\begin{table}[!htb] \caption{Keywords} \label{tab:keywords}
  \begin{center}
    \begin{tabular}[c]{p{7em}|p{13em}} {\textbf{Focus}} & microservices \\
      \hline \textbf{Refactoring} & {migration, decomposition, identify, refactor, evolve, discover, transition } \\
      \hline \textbf{Target} & monolith \\
      \hline \textbf{Outcome} & approach, tool \\
    \end{tabular}
  \end{center}
\end{table}

\subsubsection*{First iteration} \label{subsub:first-iteration}

The first iteration of the review has the main purpose of determining how many
tools exist that are able to solve this research question or, at the very
least, help partially with it. In order to do this, one could not be limited to
tools that are documented in academic databases therefore, in addition to the
databases mentioned in \Cref{tab:databases},
\href{https://github.com}{GitHub}, \href{https://gitlab.com}{GitLab} and even
\href{https://duckduckgo.org}{DuckDuckGo} were searched for, even though they
do not represent a scientific search engine.

By using some keywords mentioned in \Cref{tab:keywords} the following
initial trial query was created:

\begin{center}
  \emph{(``microservice" OR ``micro-service") AND (``migration" OR
  ``identification") AND (``monolithic" OR ``monolith") AND (``tool")}
\end{center}

\subsubsection*{Second iteration} \label{subsub:second-iteration}

In the second iteration of the review process, the focus is on locating
publications that describe alternative approaches for migrating from monolithic
to microservices architectures that may not have been implemented in practice.
This will allow an increased understanding of the current state of the art in
this area, identify any gaps or areas where further research is needed, and
determine what can be improved upon. This information will be useful in guiding
the development of our tool and abstraction.

As this was the beginning of a new iteration of the review process, the results
of the previous ``test run" were not considered when extracting literature.
Only after this iteration was completed did we incorporate the previously
analysed literature and exclude it from further analysis to avoid duplication
of effort. This allowed us to focus on identifying new and potentially relevant
publications for our purposes.

Relying on the keywords identified in \Cref{tab:keywords}, the following
query was created:

\begin{center}
  \emph{(microservice* OR micro?service*) AND (migrat* OR identif*) AND
  (monolith*) AND (migrat* NEAR/2 (process* OR approach*))}
\end{center}

In the event that the publications located through the snowballing approach are
no longer adding significant value to the overall understanding of the topic, a
new search query should be applied to the remaining databases. In some
instances, the query produced more than two thousand results, which would have
been impractical to analyse within the given timeframe. Therefore, we modified
the query to focus only on the titles and abstracts of the publications. The
revised query that should be used is:

\begin{center}
  \emph{(microservice* OR "micro-service") AND (migrat* OR decompos* OR
  identif* OR refactor* OR evolv* OR extract* OR discover* OR transition*)}
\end{center}
