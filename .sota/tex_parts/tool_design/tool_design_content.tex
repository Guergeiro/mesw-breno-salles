During our analysis described in \Cref{sub:publication-analysis}, locating any
relevant works that could address the second research question (RQ2) was
impossible. As a result, it will be necessary to tackle this question and
strive to answer it.

In order to address RQ2, we intend to develop an application with several
functionalities. The functionalities we aim to include should answer a specific
problem.

\subsubsection*{Problem}

\textit{Multiple tools exist, but how do we present them for the user to
choose?}

We create a list of all available tools that users can select which one they
want to use for a decomposition task.

\subsubsection*{Problem}

\textit{Each tool accepts a language that might not be the same to others, how
can we show and handle this information?}

In the tool list we also show the languages compatible with it. Upon selecting
one of them, you can only select more if they share a common language.

\subsubsection*{Problem}

\textit{Tools may also have a parameters that may be tuned to optmise the
decomposition output, therefore we should allow users to tune them.}

For tools that are selected, and if the tool allows parameter tuning, we allow
users to fill in the parameter values.

\subsubsection*{Problem}

\textit{For each of the tools selected, one or multiple decompositions may be
produced, how to compare them efficiently?}

Give a interface where users are able to select the decompositions they want to
compare, toggle to view more information of each specific decomposition,
evaluate the relationship between each microservice in the cluster of
microservices that make the decomposition.

In \Cref{sub:requirements}, we gather some functional requirements that break
down system features and functions as well as some non-functional requirements
that determine how the system will implement these features. Some functional
requirements were derived of the problems stated. \Cref{sub:interface_overview}
contains some mockups of a possible frontend interface.
