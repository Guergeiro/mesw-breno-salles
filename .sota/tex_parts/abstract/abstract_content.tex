In the world of software development, the concept of microservices is gaining
popularity. This architectural style has received much attention in both
business and academia and converting a monolithic application into a
microservice-based application has become a regular practice. Companies with
limited resources struggle with migrating their existing monolithic
applications to microservices, and software architects and developers
frequently face challenges due to a lack of complete awareness of alternative
migration methodologies, making the refactoring process even harder.

This dissertation aims to structurally analyse the state of the art in
migrating monolithic applications to microservices architectural style, mainly
how tools are helping architects, engineers, and developers in this migration
and how automated they are. A systematic literature review identified one
hundred and six relevant publications. These publications were organised and
grouped to provide a more comprehensive understanding of the current tools
available for microservice migration.

Furthermore, we present an extensible application to help architects,
engineers, and developers during the refactoring process by addressing gaps in
understanding various migration tools and approaches, allowing for easy
comparison between multiple options. This application was then subjected to an
empirical study to extract usability and workload metrics using System
Usability Scale and Raw Task Load Index. The results proved positive for
usability and mixed for workload performance.
