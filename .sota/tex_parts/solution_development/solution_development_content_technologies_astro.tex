One of the first decisions that we have considered is which target platform the
tool should support. There are three major platforms that are relevant for this
purpose: macOS, Windows, and Linux. Determining which of these platforms to
focus on would require extensive surveying to ensure that the tool will be used
by the intended audience. While market share data suggests that Windows is the
most widely used desktop operating system, followed by macOS and Linux
\citeWEB{desktop-usage-worldwide}, this may not necessarily reflect the
operating systems used by the architects, engineers, and developers who are
responsible for migration. Given the limited time frame, it is infeasible to
conduct a thorough survey to determine the preferred operating system of this
target audience. As a result, we have decided to adopt a more cross-platform
solution. After evaluating the options, we have determined that the browser is
the most suitable platform for this purpose.

One of the most difficult decisions of a frontend developer is choosing between
the UI framework; this is primarily due to the difficulty in switching once a
particular framework is adopted. Astro addresses this by adopting a
framework-agnostic approach \citeWEB{astro}. It enables the creation of components
in popular frameworks such as React, Svelte, Vue, Solid, among others, which
translates into being unrestricted by the technological specifics of each
framework, thereby enabling the integration of components from various
ecosystems to speed up the development process.

React, also known as React.js or ReactJS, is a popular frontend library
utilised for constructing component-based user interfaces \citeWEB{react-ui}.
Developed by Meta (formerly Facebook), React enables the development of
single-page applications by providing a framework for building reusable and
modular components.

Solid is a JavaScript framework developed by Ryan Carniato \citeWEB{solid},
explicitly designed for creating interactive web applications. It empowers
developers to leverage their existing knowledge of HTML and JavaScript to
construct reusable components that can be utilised across the entire
application. It shares most of React's functionality but heavily focuses on
bundle size and performance \citeWEB{solid}.
