Redis is an in-memory multi-model database famous for its sub-millisecond
latency. It was created in 2009 by Salvatore Sanfilippo based on the idea that
a cache can also be a durable data store. Around this time, apps like Twitter
were growing exponentially and needed a way to deliver data to their end users
faster than a relation database could handle.

Redis, which means Remote Dictionary Server \cite{redis-meaning}, was adopted
by some of the most trafficked sites in the world, because it changed the
database game by creating a system in which data is always modified or read
from the main computer memory as opposed to the much slower disk, but at the
same time, it stores its data on the disk so it can be reconstructed as needed.

Every data point in the database is a key followed by a value that can be any
of its many data structures, like lists, sets, streams, json, an others
\cite{redis-data-types}. It can be used as a distributed key-value store,
cache, and message broker \cite{redis-usage}. This latter use case is exactly
how Redis is used in the context of this application, by serving as the
communication mechanism between the backend and the underlying tools. In the
case of the implementation of the underlying tool, Redis is used as an internal
event queue.
