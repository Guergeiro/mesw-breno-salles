The authentication and authorisation mechanisms implemented within the system
serve multiple purposes. It allows users to retrieve their previous
decompositions and separate their own findings and knowledge from those of
others. This ensures a personalised and enhanced user experience when returning
to the application (FR07).

Authentication, as a standalone concept, is not explicitly implemented. Users
can be created by making a POST request to the \textit{/users} endpoint, which
generates a new user without requiring any additional fields.

On the other hand, authorisation is enforced for specific paths within the API.
The paths \textit{/results} and their subpaths, as well as
\textit{/decompositions} and their subpaths, require a user id to be
accessible. This mechanism ensures that each user's content remains segregated,
preventing mixing or unauthorised access to other users' data.

To provide a seamless user experience, the API client (e.g., the frontend
application described in \Cref{sub:frontend}) needs to store the user id
locally. This enables smooth integration when the user returns to the
application, allowing them to retrieve their personalised data and settings
effortlessly.

Further implementation details of the frontend application can be found in
\Cref{sub:frontend_auth}, providing a comprehensive overview of how the
frontend interacts with the backend API to deliver the desired functionality
and user experience.
