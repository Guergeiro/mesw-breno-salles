The authentication and authorization mechanism serve as a way of enabling the user to come back with its user identification and retrieve previous decompositions, while being also separating its own findings and knowledge from others, providing a better user experience when coming back to the application.
Authentication per se does not exist. A user can be created by simply making a \textit{POST} request to the \textit{/users} endpoint in order to retrieve a new user and no extra fields are required. As for authorization, the paths \textit{/results} and subpaths as well as \textit{decompositions} and subpaths require a user id to be accessible, making sure users content is not mixed together.
To provide a better user experience, a client of this API, for example the frontend of this solution as described in \Cref{sub:frontend}, will need to save the user id locally and provide a seamless integration upon the user coming back to the application. The implementation details of the frontend are detailed in \Cref{sub:frontend_auth}.
