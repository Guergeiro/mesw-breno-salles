The core of the solution is on the comparison page, where the users:

\begin{enumerate}
  \item Are able to toggle a decomposition to be visible or not.
  \item Can visualise all the microservices generated by each decomposition.
  \item Can focus on a microservice and check the modules that make each service.
  \item Are able to look at which modules are in each microservice across different decompositions.
\end{enumerate}

During the evaluation of how to build the the comparison view, multiple tools that present visualisation data were considered. We end up focusing on four that caught our eye, Graphviz \footnote{\url{https://graphviz.org/}}, D3.js \footnote{\url{https://d3js.org/}}, Gephi \footnote{\url{https://gephi.org/}} and Chart.js \footnote{\url{https://chartjs.org/}}. After deeply analysing the pros and cons of each, we decided to create a matrix to better understand each in four categories, \textit{Customizability}, \textit{Ease of use}, \textit{Charts available} and \textit{Real-time interactivity}.

\begin{table}[!htb] \caption{Visualisation Tool Comparison} \label{tab:visualisation_tool_comparison}
  \begin{center}
    \begin{tabular}[c]{p{8em}|p{8em}|p{8em}|p{8em}|p{8em}}
      \textbf{Visualisation Tool} &
      \textbf{Customizability} &
      \textbf{Ease of use} &
      \textbf{Charts available} &
      \textbf{Real-time interactivity} \\
      \hline Graphviz &
        Limited customization options, suitable for creating static diagrams and graphs. &
        Easy-to-use syntax, accessible to non-experts. &
        Suitable for creating static diagrams and graphs, with limited support for charts. &
        Limited interactivity options. \\
      \hline D3.js &
        Highly customizable, suitable for creating custom and interactive visualizations. &
        Steep learning curve, requires a good understanding of JavaScript and web technologies. &
        Suitable for creating a wide range of charts, including bar charts, line charts, scatterplots, and more. &
        Provides advanced interactivity options, such as brushing and zooming, making it suitable for real-time visualizations. \\
      \hline Gephi &
        Customizable with a range of features, but may have limited design options. &
        User-friendly interface, easy to learn for beginners. &
        Suitable for creating network graphs and visualizations. &
        Allows users to interact with graphs and manipulate layouts in real-time. \\
      \hline Chart.js &
        Customizable with various options for color schemes, font styles, and animation effects. &
        Simple and easy-to-use API, minimal setup required. &
        Supports common chart types such as line charts, bar charts, pie charts, and more. &
        Provides basic interactivity features such as hover effects and tooltips. \\
    \end{tabular}
  \end{center}
\end{table}

Taking a look at what each tool can offer as presented in \Cref{tab:visualisation_tool_comparison} and taking into account that the majority of the industry leans towards node graphs for presenting microservices TODO: https://doi.org/10.1109/SOSE55356.2022.00011 , D3.js was concluded to be the best and most powerful visualisation tool for our use case.

Although having a powerful tool is important, knowing how to present the information is just as important, meaning the shapes of the visualisation tool are relevant. As mentioned by Daniel Moody (TODO: citation), one of the best ways of expressing variation between entities is by using different shapes, different colours, and relationships between entities using various strokes and line dashes that carry weight while the user is looking at the visualisation.

\begin{table}[!htb] \caption{Visual Expressiveness} \label{tab:visual_expressiveness}
  \begin{center}
    \begin{tabular}[c]{p{8em}|p{8em}|p{8em}|p{8em}}
      \textbf{Entity} &
      \textbf{Shape} &
      \textbf{Size} &
      \textbf{Colour} \\
      \hline Service & Circle & Variable according amount of modules & Based selected decomposition * \\
      \hline Module & Square & Static & Mix between selected decompositions \\
      \hline Relationships & Line & Static & Static \\
    \end{tabular}
  \end{center}
 \vspace{1ex}
 {\raggedright \textit{* Each decomposition colour is random.} \par}
\end{table}

\Cref{tab:visual_expressiveness} shows how the visualisation tool expressed each different entity.


