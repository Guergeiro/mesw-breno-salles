The comparison page serves as the core component of the solution, providing
users with essential functionalities. On this page, users can:

\begin{enumerate}
  \item Toggle the visibility of decompositions allowing users to focus on the
    ones they want to compare.
  \item Visualise microservices generated by each decomposition, enabling users
    to gain insights into the composition and structure of the system
    components.
  \item Focus on a microservice and check its constituent modules where users
    can delve into the details of each microservice and understand its internal
    components.
  \item Compare modules across different decompositions, which helps users
    identify similarities, differences, and variations in the composition of
    the system components.
\end{enumerate}

During the evaluation of various tools for visualising the comparison view,
four options were considered: Graphviz \footnote{\url{https://graphviz.org/}},
D3.js \footnote{\url{https://d3js.org/}}, Gephi
\footnote{\url{https://gephi.org/}}, and Chart.js
\footnote{\url{https://chartjs.org/}}. Each tool was assessed based on four
categories: Customizability, Ease of use, Charts available, and Real-time
interactivity. A matrix analysis was conducted (as presented in
\Cref{tab:visualisation_tool_comparison}) to assess the strengths and
weaknesses of each tool.

\begin{table}[!htb] \caption{Visualisation Tool Comparison} \label{tab:visualisation_tool_comparison}
  \begin{center}
    \begin{tabular}[c]{p{4em}|p{8em}|p{8em}|p{8em}|p{8em}}
      \textbf{Visuali\-sation Tool} &
      \textbf{Customizability} &
      \textbf{Ease of use} &
      \textbf{Charts available} &
      \textbf{Real-time interactivity} \\
      \hline Graphviz &
        Limited customization options, suitable for creating static diagrams and graphs. &
        Easy-to-use syntax, accessible to non-experts. &
        Suitable for creating static diagrams and graphs, with limited support for charts. &
        Limited interactivity options. \\
      \hline D3.js &
        Highly customizable, suitable for creating custom and interactive visualizations. &
        Steep learning curve, requires a good understanding of JavaScript and web technologies. &
        Suitable for creating a wide range of charts, including bar charts, line charts, scatterplots, and more. &
        Provides advanced interactivity options, such as brushing and zooming, making it suitable for real-time visualizations. \\
      \hline Gephi &
        Customizable with a range of features, but may have limited design options. &
        User-friendly interface, easy to learn for beginners. &
        Suitable for creating network graphs and visualizations. &
        Allows users to interact with graphs and manipulate layouts in real-time. \\
      \hline Chart.js &
        Customizable with various options for color schemes, font styles, and animation effects. &
        Simple and easy-to-use API, minimal setup required. &
        Supports common chart types such as line charts, bar charts, pie charts, and more. &
        Provides basic interactivity features such as hover effects and tooltips. \\
    \end{tabular}
  \end{center}
\end{table}

Considering the industry's preference for node graphs in presenting
microservices \cite{cerny2022microservice} and after careful consideration of
the pros and cons, D3.js was determined to be the most powerful and suitable
visualisation tool for the specific use case.

In addition to having a powerful tool, the way information is presented and the
visual elements used are crucial. As suggested by Moody
\citeauthor{moody2009physics}, using different shapes, colours, strokes, and line
dashes to depict entities and their relationships is an effective way of
expressing variation between entities in visualisations.


\begin{table}[!htb] \caption{Visual Expressiveness} \label{tab:visual_expressiveness}
  \begin{center}
    \begin{tabular}[c]{p{8em}|p{8em}|p{8em}|p{8em}}
      \textbf{Entity} &
      \textbf{Shape} &
      \textbf{Size} &
      \textbf{Colour} \\
      \hline Service & Circle & Variable according amount of modules & Based selected decomposition * \\
      \hline Module & Square & Static & Mix between selected decompositions \\
      \hline Relationships & Line & Static & Static \\
    \end{tabular}
  \end{center}
 \vspace{1ex}
 {\raggedright \textit{* Each decomposition colour is random.} \par}
\end{table}

\Cref{tab:visual_expressiveness} illustrates how each visual representation
tool expresses different entities, highlighting the specific shapes and visual
cues employed to convey information effectively.


