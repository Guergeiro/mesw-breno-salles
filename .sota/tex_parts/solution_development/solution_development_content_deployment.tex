Ideally, each component of the system is  deployed as a microservice in a
Docker container to facilitate scalability and cross-platform deployment. This
is especially important for the tool runtime component, as it is uncertain
whether the tools will take advantage of multithreading. By using Docker, we
can create a new container for each job, which would allow us to handle
multiple jobs concurrently and avoid the need for users to wait for their jobs
to complete. Of course, it is possible that certain tool limitations may
prevent us from implementing this approach, but it is our goal to use
microservices and Docker to the greatest extent possible to ensure the
flexibility and scalability of the tool.

The application incorporates a Continuous Integration and Delivery (CI/CD)
process. Whenever a new push is made to the \textit{master} branch of the Git
repository, it triggers a GitHub Action that initiates the build process and
deploys the applications. The hosting platform used for the application is
Railway\footnote{\url{https://railway.app/}}. However, since the application is
packaged with Docker, it can be hosted on any platform that supports Docker
images, as explained in \Cref{sub:docker}. This flexibility allows the
application to be deployed to various hosting environments based on specific
requirements and preferences.
