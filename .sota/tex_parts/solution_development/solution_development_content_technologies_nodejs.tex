Node.js is an open-source runtime environment that enables the execution of
JavaScript code outside of the web browser. Initially developed in 2009 by Ryan
Dahl as a response to the criticisms he had expressed towards the widely used
Apache HTTP Server \citeWEB{ryan-dahl-nodejs}.

In contrast to traditional execution models, where code is executed
sequentially and relies on thread mechanisms to prevent blocking, Node.js
adopts a distinct approach. It capitalises on JavaScript's event loop
\citeWEB{javascript-event-loop} paradigm to manage asynchronous I/O operations.
This event-driven architecture enables Node.js to handle concurrent requests
efficiently, maximising performance \citeWEB{nodejs-event-loop}.

Moreover, the Node.js ecosystem offers modules via
npm\footnote{\url{https://www.npmjs.com/}} that address various common
challenges encountered in software development. These modules help with
functionalities such as file and network access, manipulation of binary data,
cryptography, and other general-purpose tasks.

Node.js has become widely adopted as a versatile tooling platform for
developing a wide range of applications, even when those applications
themselves do not run on Node.js. Notably, frameworks such as
React\footnote{\url{https://github.com/facebook/react}},
Vue\footnote{\url{https://github.com/vuejs/core}}, and
Angular\footnote{\url{https://github.com/angular/angular}} utilise Node.js for
tooling purposes, specifically for compiling their JavaScript framework code
into browser-compatible JavaScript.

In the context of this work, Node.js is used for deploying both the backend and
the underlying tool system, while is also used as a build tool for the
frontend.
