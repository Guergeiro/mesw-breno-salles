Kamil Myśliwiec created NestJS to facilitate the development of efficient and
scalable backend applications using Node.js. This framework supports JavaScript
and TypeScript languages, allowing developers to choose the language that best
suits their project requirements.

One of the distinguishing features of NestJS is its ability to use various
programming paradigms, namely Functional Programming (FP), Object-Oriented
Programming (OOP), and Functional Reactive Programming (FRP). By incorporating
elements from these paradigms, NestJS provides developers with a toolkit to
build applications using a combination of functional and object-oriented
concepts that allow them to leverage the strengths of each approach, fostering
code modularity, maintainability, and reusability \cite{nestjs}.

NestJS is an opinionated framework that draws inspiration from Angular and is
similar to the Spring framework for Java in the Node.js ecosystem. Like Spring,
NestJS offers comprehensive documentation outlining solutions to common backend
challenges. It accomplishes this by providing adapters and integrations with
popular existing solutions. With the speed, ease of use and integrations it
provides, NestJS is used for both the backend and the underlying tool
implementation.
